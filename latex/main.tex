\documentclass[a4paper, 11pt]{article}

\usepackage{geometry}
\geometry{paper=a4paper,top=2.5cm,bottom=2.5cm,right=2.5cm,left=2.5cm}

\usepackage[french]{babel}

\usepackage[utf8]{inputenc}
\usepackage[T1]{fontenc}

\usepackage[parfill]{parskip}

\usepackage{siunitx}

\title{\textbf{\textsc{INFO0054-1} : Programmation fonctionnelle}\\Projet : Puzzles réguliers}
\author{Maxime \textsc{Meurisse} (s161278)\\François \textsc{Rozet} (s161024)\\}
\date{\today}

\begin{document}
    \maketitle
    \section{Solver}
    Notre solver parcourt l'arbre des chemins sans cycles dans l'ordre dicté par une \emph{recherche en largeur}. Il est donc assuré de trouver toutes les solutions et dans l'ordre désiré, c.-à-d. de la plus courte à la plus longue.
    \setcounter{section}{2}
    \section{Heuristique}
    Le solver \og{}avec heuristique\fg{} implémente une \emph{priority queue} dont la priorité\footnote{La plus petite valeur est prioritaire.} est l'heuristique au dernier état du chemin.
    Nous avons implémenté les deux heuristiques de taquin $H_1$ et $H_2$. En mesurant leur temps d'exécution pour trouver un grand ($\num{300}$) nombre de solutions, il est apparu que la seconde était un peu plus rapide ($\SI{46}{\second}$ contre $\SI{21}{\second}$). C'est donc cette dernière que nous avons désignée comme heuristique.
\end{document}
