\documentclass[a4paper, 11pt]{article}

\usepackage{geometry}
\geometry{paper=a4paper,top=2.5cm,bottom=2.5cm,right=2.5cm,left=2.5cm}

\usepackage[french]{babel}

\usepackage[utf8]{inputenc}
\usepackage[T1]{fontenc}

\usepackage[parfill]{parskip}

\usepackage{siunitx}

\title{\textbf{\textsc{INFO0054-1} : Programmation fonctionnelle}\\Projet : Puzzles réguliers}
\author{Maxime \textsc{Meurisse} (s161278)\\François \textsc{Rozet} (s161024)\\}
\date{\today}

\begin{document}
    \maketitle
    \section{Solver}
    Notre solver se base sur l'algorithme BFS. Ce dernier ne permet pas d'obtenir toutes les solutions d'un problème régulier, seulement les plus courtes, c.-à-d. celles qui ont été les \og{}premières\fg{} à visiter leurs états. \par
    Dans la méthode originale, à chaque étape, les chemins dont le dernier état a déjà été visité précédemment sont supprimés. Dans notre implémentation, nous avons décidé de les garder en mémoire. Lorsque une solution est trouvée par l'algorithme, nous réintroduisons tous les chemins dont l'état final est un état intermédaire de la solution. \par
    De cette façon nous sommes assurés de trouver toutes les solutions et dans l'ordre désiré, c.-à-d. de la plus courte à la plus longue. \par
    Néanmoins, cette méthode est loin d'être optimale. Notamment, elle ne réutilise pas la solution trouvée pour en former de nouvelles à partir des chemins mis de côté. En effet, puisque certains de ces derniers ont un état commun avec la solution, il devrait être possible d'en construire une autre au minimum aussi longue.
    \setcounter{section}{2}
    \section{Heuristique}
    Le solver \og{}avec heuristique\fg{} implémente une \emph{priority queue} dont la priorité est la somme de l'heuristique au dernier état du chemin et de la longueur de ce dernier. L'addition de la longueur permet de privilégier les chemins courts. \par
    Nous avons implémenté les deux heuristiques de taquin $H_1$ et $H_2$. En mesurant leur temps d'exécution pour trouver un grand ($\num{100}$) nombre de solutions, il est apparu clairement que la seconde était plus rapide ($\SI{15}{\second}$ contre $\SI{2000}{\second}$) . C'est donc cette dernière que nous avons désignée comme heuristique.
\end{document}
